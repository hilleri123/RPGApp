% Кодировка/язык — по желанию, если pdfLaTeX:
\usepackage{iftex}

\ifPDFTeX
  \usepackage[T2A]{fontenc}
  \usepackage[utf8]{inputenc}
  \usepackage[russian]{babel}
\else
  \usepackage{fontspec}
  \setmainfont{DejaVu Serif} % или любой шрифт с кириллицей
  \setsansfont{DejaVu Sans}
  \setmonofont{DejaVu Sans Mono}
  \usepackage[russian]{babel}
\fi

\usepackage[russian]{babel}

% Картинки
\usepackage{graphicx}
\graphicspath{{assets/images/}{assets/screenshots/}} % trailing slash обязателен 

% Ссылки в PDF (кликабельные ref, toc)
\usepackage{hyperref}

% Удобные ссылки на секции/рисунки (не обязательно)
\usepackage[nameinlink]{cleveref}

% (Опционально) якоря/ссылки на конкретные места
% \usepackage{bookmark}
\newcounter{userstory}[chapter]
\renewcommand{\theuserstory}{US-\arabic{userstory}}

\newcommand{\userstory}[2]{%
  \refstepcounter{userstory}%
  \subsection{\theuserstory: #1}%
  \label{#2}%
}


% --- Navigation doc macros ---
\newcommand{\route}[3]{% title, url, label
  \section{#1}
  \label{#3}
  \paragraph{URL} \texttt{\detokenize{#2}}
}

\newcommand{\component}[2]{% name, label
  \subsection{Компонент: #1}
  \label{#2}
}

\newcommand{\action}[2]{% name, label
  \subsubsection{Действие: #1}
  \label{#2}
}

% Плейсхолдер для скрина (чтобы не ломать сборку, пока файлов нет)
\newcommand{\screenshotPlaceholder}[1]{% what
  \paragraph{Скриншот}
  (todo) #1.
}


% --- UI link/action macros (for Navigation chapter) ---
\newcommand{\uilink}[3]{% text, href, label
  \paragraph{Ссылка: #1}
  \label{#3}
  \textbf{URL:} \texttt{\detokenize{#2}}
}

\newcommand{\uibutton}[2]{% text, label
  \paragraph{Кнопка: #1}
  \label{#2}
}


\newcommand{\screenshot}[2]{%
  \begin{figure}[htbp]
    \centering
    \includegraphics[width=\linewidth]{#1}
    \caption{#2}
  \end{figure}
}