\section{Модель данных и владение}\label{sec:rules-and-plugins:data-ownership}

\paragraph{Два контура данных}
В проекте различаются два типа данных: (1) ``нарратив'' — человекочитаемые поля сценария/сцены/сущностей; (2) ``правила'' — структурированные данные, хранящиеся в \texttt{data: json}. 

\paragraph{Владелец \texttt{data: json}}
Семантика \texttt{data: json} принадлежит плагину: только он определяет схему, дефолты, миграции и правила валидации, а также то, как эти данные интерпретируются в процессе игры. 

\paragraph{Владелец нарратива}
Семантика нарративных полей принадлежит базовому приложению (сайту и его бэкенду): их можно редактировать без подключения плагина и без знания правил. 

\paragraph{Следствие}
Сайт не должен выполнять бизнес-валидацию \texttt{data: json} и не должен пытаться ``понимать'' или частично модифицировать её структуру; он может лишь хранить и передавать этот JSON и отображать UI, который предоставил плагин. 
