\section{Зачем нужны плагины правил}\label{sec:rules-and-plugins:architecture}

\paragraph{Идея}
Сайт является хостом и прослойкой между UI и движком правил: он отображает формы, маршрутизирует запросы и сохраняет ``нарративные'' поля, но не знает предметной структуры правил. [file:232]

\paragraph{Роль плагина}
Плагин правил — единственный компонент, который понимает и валидирует доменную структуру в колонке \texttt{data: json} и может безопасно изменять её содержимое (например, ``скиллы'', параметры, вычисляемые значения). [file:232]

\paragraph{Граница ответственности}
Нарративные поля (например, \texttt{name}, \texttt{description}) редактируются в рамках заполнения сценария и не зависят от правил; правила же живут в \texttt{data: json} и управляются плагином независимо от нарратива. [file:232]
