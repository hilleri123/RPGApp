\section{Контракты плагинов}\label{sec:rules-and-plugins:contracts}

\subsection{Контракт фронтенд-плагина}\label{subsec:rules-and-plugins:contracts:frontend}

\paragraph{Назначение}
Фронтенд-плагин предоставляет UI-описание (``форму'') для редактирования \texttt{data: json} и, опционально, виджеты для отображения игровых сущностей в сессии. [file:232]

\paragraph{Минимальные возможности фронтенд-плагина}
\begin{itemize}
\item Декларировать схему/описание формы для \texttt{data: json} (поля, типы, подписи, подсказки).
\item Декларировать правила отображения (скрытие/зависимости полей) на основе текущего \texttt{data}.
\item (Опционально) предоставить UI-компоненты для экранов сессии, например вкладки/панели в правом блоке сущностей \ref{cmp:session:right-panel}. [file:232]
\end{itemize}

\paragraph{Требование к фронтенду хоста}
Хост-сайт обязан отрисовать форму плагина, отправлять изменения в API хоста и показывать результат/ошибки, но не обязан знать смысл полей внутри \texttt{data}. [file:232]

\subsection{Контракт бэкенд-плагина}\label{subsec:rules-and-plugins:contracts:backend}

\paragraph{Назначение}
Бэкенд-плагин является единственным валидатором и модификатором \texttt{data: json} и реализует ``движок'' правил. [file:232]

\paragraph{Минимальные возможности бэкенд-плагина}
\begin{itemize}
\item Валидация входящих патчей/версий \texttt{data: json}.
\item Нормализация и заполнение значений по умолчанию.
\item Миграции \texttt{data} между версиями схемы.
\item Исполнение правил на событиях сессии и генерация эффектов/сообщений.
\end{itemize}

\paragraph{Требование к бэкенду хоста}
Бэкенд хоста предоставляет единое API, через которое фронт общается с плагином, и маршрутизирует запросы в бэкенд-плагин; при этом хост не внедряет свою интерпретацию \texttt{data}. [file:232]

\subsection{Единая точка входа (API хоста)}\label{subsec:rules-and-plugins:contracts:host-api}

\paragraph{Зачем это нужно}
Даже если плагин имеет свой собственный бэкенд, взаимодействие происходит ``сквозь'' API сайта: это позволяет централизовать авторизацию, аудит и логирование. [file:232]

\paragraph{Наблюдаемость в UI}
Исполнение правил и результат действий удобно трассировать в нижнем блоке \ref{cmp:session:logs} (вкладки ``Действия/Логи/Уведомления''). [file:232]
