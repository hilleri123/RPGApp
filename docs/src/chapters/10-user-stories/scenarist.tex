\section{Сценарист}
\label{sec:us-scenarist}

Сценарист отвечает за подготовку материалов сценария: персонажи, места, карты, предметы и действия, а также связи между ними.
В интерфейсе эта роль пересекается с ролью мастера-сценариста (см. \ref{sec:us-master}).

\userstory{Подготовка контента сценария}{us:scenarist:prepare-content}

Я, как сценарист, хочу удобно создавать и редактировать элементы сценария (персонажей, места, карты, предметы и действия), а также задавать связи между ними,
чтобы формировать целостные сценарии и быстро вносить правки по мере подготовки.

\paragraph{Условия начала}
У меня есть доступ к разделу сценариев, и я могу открыть существующий сценарий или создать новый.

\paragraph{Основной сценарий}
\begin{enumerate}
  \item Открываю сценарий и перехожу к редактированию контента.
  \item Создаю или редактирую сущности (персонажи, локации, предметы, действия).
  \item Добавляю связи: где находится персонаж, какие предметы доступны в локации, какие действия доступны при выполнении условий.
  \item Проверяю, что элементы отображаются и связаны ожидаемым образом.
  \item Сохраняю изменения.
\end{enumerate}

\paragraph{Ожидаемый результат}
Контент сценария сохранён, связи между элементами восстановимы и используются далее при запуске и ведении игры.
