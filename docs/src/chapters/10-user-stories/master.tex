\section{Мастер}
\label{sec:us-master}

Мастер работает в двух режимах: подготовка сценария (как мастер-сценарист) и ведение запущенной игры (как мастер ведущий).
Ниже описаны ключевые пользовательские истории и ожидаемое поведение системы.

\userstory{Создание сценария}{us:m:create-scenario}

Я, как мастер-сценарист, хочу удобно создавать персонажей, места, карты, предметы и действия, а также связи между ними,
чтобы формировать целостные сценарии.

\paragraph{Критерий готовности}
Я могу создать базовый набор элементов сценария, связать их и сохранить, не переходя во внешний инструмент.

\userstory{Запуск сценария}{us:m:start-scenario}

Я, как мастер, хочу иметь возможность выбрать готовый сценарий и запустить его со старта, даже если уже открыт другой сценарий,
чтобы иметь возможность быстро перезапустить сценарий.

\paragraph{Ожидаемое поведение}
Запуск приводит игру в начальное состояние выбранного сценария.
Текущий открытый сценарий или активная игра не мешают выбору и запуску.

\userstory{Набор игроков}{us:m:lobby}

Я, как мастер, хочу видеть подключенных игроков и состояние их настройки окружения (цвет, имя, персонаж и т.д.),
чтобы понимать готовность игроков к началу игры.

\paragraph{Критерий готовности}
Я вижу список игроков и их статус готовности; изменения статусов обновляются без ручной перезагрузки страницы.

\userstory{Ведение сценария}{us:m:run-scenario}

Я, как мастер начатой игры, хочу видеть текст текущего действия сценария и иметь быстрый доступ к тексту смежных действий,
чтобы минимизировать ошибки, связанные с нелинейной подачей нарратива.

\paragraph{Критерий готовности}
Я могу быстро перейти к смежным действиям и вернуться назад, не теряя контекст текущей сцены/действия.

\userstory{Действие сценария}{us:m:prepared-action}

Я, как мастер начатой игры, хочу иметь быструю возможность разыграть подготовленное в сценарии действие,
чтобы вести игру динамично.

\paragraph{Критерий готовности}
Я запускаю выбранное подготовленное действие, и оно становится “текущим” для игроков (в пределах их доступа).

\userstory{Импровизированное действие}{us:m:improvised-action}

Я, как мастер начатой игры, хочу иметь быструю возможность создать и разыграть неподготовленное в сценарии действие,
чтобы реагировать на нестандартные решения игроков.

\paragraph{Критерий готовности}
Я могу создать действие “на лету” и немедленно применить его в текущем ходе игры.

\userstory{Состояние персонажей игроков}{us:m:pc-state}

Я, как мастер начатой игры, хочу постоянно видеть состояния персонажей игроков,
чтобы иметь постоянный контроль над происходящим.

\paragraph{Критерий готовности}
Я вижу состояния всех персонажей игроков в одном месте и замечаю изменения без ручных обновлений.

\userstory{Создание сцены}{us:m:create-scene}

Я, как мастер начатой игры, хочу иметь возможность быстро создавать сцену для игроков из элементов игры (предметы, персонажи, места),
чтобы точнее передавать игрокам ситуацию.

\paragraph{Критерий готовности}
Я собираю сцену из существующих элементов и задаю, что именно видят разные игроки/персонажи.

\userstory{Заметки для игроков}{us:m:notes}

Я, как мастер начатой игры, хочу иметь возможность заготовленный текст отсылать игрокам с гранулярным доступом,
чтобы давать скрытые вводные и распределять информацию между персонажами.

\paragraph{Критерий готовности}
Я создаю заметку, выбираю, кому она видна, и отправляю её адресатам.
