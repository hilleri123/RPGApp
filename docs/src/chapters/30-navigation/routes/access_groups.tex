\route{Доступ: группы и пользователи}{/access_groups}{url:access-groups}

\paragraph{Назначение}
Раздел управления доступом: группы и пользователи (назначение прав, просмотр/настройка групп, переход к настройкам конкретной группы).

\screenshotPlaceholder{Страница \texttt{/access\_groups}}

\component{Header / Top bar}{cmp:access-groups:header}

\paragraph{Назначение}
Глобальная навигация по приложению, доступ к профилю и выходу из системы.

\action{Переход по логотипу}{act:access-groups:header:logo}
\uilink{Логотип / заголовок}{/}{ui:access-groups:logo}
\paragraph{Действие}
Переход на главную страницу.

\action{Навигация: Главная}{act:access-groups:header:nav-home}
\uilink{Пункт меню ``Главная''}{/}{ui:access-groups:nav-home}
\paragraph{Действие}
Переход на главную страницу.

\action{Навигация: Правила}{act:access-groups:header:nav-rules}
\uilink{Пункт меню ``Правила''}{/rules}{ui:access-groups:nav-rules}
\paragraph{Действие}
Переход в раздел правил.

\action{Навигация: Сценарии}{act:access-groups:header:nav-scenarios}
\uilink{Пункт меню ``Сценарии''}{/scenarios}{ui:access-groups:nav-scenarios}
\paragraph{Действие}
Переход в раздел сценариев.

\action{Навигация: Группы}{act:access-groups:header:nav-groups}
\uilink{Пункт меню ``Группы''}{/access_groups}{ui:access-groups:nav-groups}
\paragraph{Действие}
Переход в текущий раздел (управление доступом).

\action{Переход: Профиль}{act:access-groups:header:nav-me}
\uilink{Профиль пользователя}{/me}{ui:access-groups:nav-me}
\paragraph{Действие}
Переход на страницу профиля текущего пользователя.

\action{Выход}{act:access-groups:header:logout}
\uibutton{Кнопка ``Выйти''}{ui:access-groups:btn-logout}
\paragraph{Действие}
Завершает текущую сессию пользователя и переводит в состояние ``не авторизован'' (точный редирект зависит от реализации).

\action{Открыть мобильное меню}{act:access-groups:header:mobile-menu}
\uibutton{Кнопка меню (``бургер'')}{ui:access-groups:btn-mobile-menu}
\paragraph{Действие}
Открывает/закрывает мобильную навигацию (на малых экранах).

\component{Вкладки раздела}{cmp:access-groups:tabs}

\paragraph{Назначение}
Переключение между представлениями ``Группы'' и ``Пользователи'' внутри одного раздела.

\action{Открыть вкладку ``Группы''}{act:access-groups:tab-groups}
\uibutton{Вкладка ``Группы''}{ui:access-groups:tab-groups}
\paragraph{Действие}
Показывает панель управления группами доступа.

\action{Открыть вкладку ``Пользователи''}{act:access-groups:tab-users}
\uibutton{Вкладка ``Пользователи''}{ui:access-groups:tab-users}
\paragraph{Действие}
Показывает панель управления пользователями (в текущем HTML панель скрыта/пустая; наполнение уточнить по реализации).

\component{Панель ``Группы''}{cmp:access-groups:groups-panel}

\paragraph{Назначение}
Работа со списком групп: поиск, создание, переход к настройкам конкретной группы.

\action{Поиск группы}{act:access-groups:groups:search}
\uibutton{Поле ``Поиск группы по названию...''}{ui:access-groups:groups:search}
\paragraph{Действие}
Фильтрует список групп по введённой строке.

\action{Добавить группу}{act:access-groups:groups:add}
\uibutton{Кнопка ``Добавить''}{ui:access-groups:groups:add}
\paragraph{Действие}
Открывает сценарий создания новой группы (форма/диалог; точную механику уточнить по реализации).

\component{Элемент списка групп}{cmp:access-groups:groups:list-item}

\paragraph{Назначение}
Представляет одну группу в списке и даёт переход к настройкам.

\action{Открыть настройки группы}{act:access-groups:groups:settings}
\uilink{Кнопка/ссылка ``Настройки'' для группы}{/access_groups/<group_id>}{ui:access-groups:groups:settings}
\paragraph{Действие}
Переходит на страницу настроек выбранной группы (детальная страница группы).

\paragraph{Пример}
В текущем состоянии есть группа \texttt{Creators} и ссылка вида:
\texttt{/access\_groups/6e7c1228-d5ec-483a-853e-9e4c22429046}.
