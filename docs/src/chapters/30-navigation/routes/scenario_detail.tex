\route{Редактирование сценария}{/scenarios/<scenario_id>}{url:scenario-detail}

\paragraph{Назначение}
Страница редактирования конкретного сценария: подготовка контента сценария (объекты, сюжет, экспозиции, локации и т.д.).
Название сценария отображается в заголовке страницы.

\paragraph{Пример URL}
\texttt{/scenarios/515d2aaa-768b-4456-a837-bb160cf473a8} (сценарий \texttt{Северный полюс}).


\component{Заголовок сценария}{cmp:scenario-detail:title}

\paragraph{Назначение}
Отображает контекст редактирования: имя сценария.

\paragraph{Текущее отображение}
В заголовке отображается строка вида \texttt{Редактирование: <name>} (в примере: \texttt{Северный полюс}).

\component{Рабочая область редактора}{cmp:scenario-detail:workspace}

\paragraph{Назначение}
Основная область редактирования сценария.

\paragraph{Состав (ожидаемый)}
Набор панелей/вкладок для управления объектами сценария (например: story beat, экспозиции сцен, локации, НПС, предметы, заметки, счётчики, эталоны).


\component{Панель: Story beat}{cmp:scenario-detail:story-beat}
\paragraph{Назначение}
Редактирование сюжетных точек (story beat): создание, порядок, тексты для мастера/игроков, связи с локациями и НПС.


\screenshot{components/scenario_story_beat.png}{Вкладка Story beat}

\component{Панель: Экспозиции}{cmp:scenario-detail:scene-exposure}
\paragraph{Назначение}
Создание и редактирование экспозиций сцен, привязка к story beat или локации, наполнение (НПС/предметы/препятствия).


\component{Панель: Локации}{cmp:scenario-detail:locations}
\paragraph{Назначение}
Управление локациями сценария, их описаниями и иерархией.


\screenshot{components/scenario_location.png}{Вкладка Локации}

\component{Панель: НПС}{cmp:scenario-detail:npcs}
\paragraph{Назначение}
Управление НПС: описания, флаги видимости, привязки к сюжетным точкам и экспозициям.


\screenshot{components/scenario_npc.png}{Вкладка НПС}


\component{Панель: Предметы}{cmp:scenario-detail:items}
\paragraph{Назначение}
Управление предметами: описания, видимость, размещение (локация/НПС/персонаж), контейнеры и владение.


\screenshot{components/scenario_item.png}{Вкладка Предметы}


\component{Панель: Заметки}{cmp:scenario-detail:notes}
\paragraph{Назначение}
Заметки мастера и управление доступом (кому видно).


\screenshot{components/scenario_note.png}{Вкладка Заметки}


\component{Панель: Счётчики}{cmp:scenario-detail:counters}
\paragraph{Назначение}
Счётчики прогресса/состояния сценария (значения, границы, привязки).


\screenshot{components/scenario_counter.png}{Вкладка Счётчики}


\component{Панель: Эталоны (фабрики)}{cmp:scenario-detail:templates}
\paragraph{Назначение}
Шаблоны/эталоны сущностей для быстрого клонирования во время сессии.
