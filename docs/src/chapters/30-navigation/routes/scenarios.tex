\route{Сценарии}{/scenarios}{url:scenarios}

\paragraph{Назначение}
Раздел управления сценариями: просмотр списка, создание нового сценария, переход к редактированию, удаление сценария.


\component{Список сценариев}{cmp:scenarios:list}

\paragraph{Назначение}
Отображает все доступные сценарии и предоставляет быстрые действия по каждому из них.

\screenshot{components/scenarios.png}{Список сценариев}

\action{Создать сценарий}{act:scenarios:create}
\uibutton{Кнопка ``Создать'' (на странице сценариев)}{ui:scenarios:btn-create}
\paragraph{Действие}
Создаёт новый сценарий или открывает форму создания (точный целевой URL/модалка уточняется по реализации).

\component{Элемент списка сценариев}{cmp:scenarios:list-item}

\paragraph{Назначение}
Одна строка/карточка сценария: название и действия управления.

\action{Открыть сценарий (редактирование)}{act:scenarios:open}
\uilink{Кнопка ``Открыть сценарий'' (иконка карандаша)}{/scenarios/<scenario_id>}{ui:scenarios:btn-open}
\paragraph{Действие}
Переходит на страницу сценария для просмотра/редактирования.

\paragraph{Примеры URL}
\texttt{/scenarios/52978346-3a5d-4a67-b39b-bb666fe70aed},
\texttt{/scenarios/515d2aaa-768b-4456-a837-bb160cf473a8}.

\action{Удалить сценарий}{act:scenarios:delete}
\uibutton{Кнопка ``Удалить'' (иконка корзины)}{ui:scenarios:btn-delete}
\paragraph{Действие}
Удаляет выбранный сценарий (как минимум требует подтверждения; поведение после удаления: обновление списка).
