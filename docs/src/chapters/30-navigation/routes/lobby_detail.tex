\route{Лобби}{/lobby/<lobby_id>}{url:lobby-detail}

\paragraph{Назначение}
Страница конкретного лобби (игровой сессии): управление участниками и выбор/сброс персонажей.

\paragraph{Пример URL}
\texttt{/lobby/4846dfdb-c4b6-41dd-aef1-ef4e7f4f66c2}.

\screenshotPlaceholder{Страница лобби}

\component{Header / Top bar}{cmp:lobby:header}

\paragraph{Назначение}
Глобальная навигация по приложению и доступ к профилю пользователя.

\action{Навигация: Главная}{act:lobby:header:nav-home}
\uilink{Пункт меню ``Главная''}{/}{ui:lobby:nav-home}
\paragraph{Действие}
Переход на главную страницу.

\action{Навигация: Правила}{act:lobby:header:nav-rules}
\uilink{Пункт меню ``Правила''}{/rules}{ui:lobby:nav-rules}
\paragraph{Действие}
Переход в раздел правил.

\action{Навигация: Сценарии}{act:lobby:header:nav-scenarios}
\uilink{Пункт меню ``Сценарии''}{/scenarios}{ui:lobby:nav-scenarios}
\paragraph{Действие}
Переход в раздел сценариев.

\action{Навигация: Группы}{act:lobby:header:nav-groups}
\uilink{Пункт меню ``Группы''}{/access_groups}{ui:lobby:nav-groups}
\paragraph{Действие}
Переход в раздел управления доступом.

\action{Переход: Профиль}{act:lobby:header:nav-me}
\uilink{Профиль пользователя}{/me}{ui:lobby:nav-me}
\paragraph{Действие}
Открывает страницу профиля пользователя.

\action{Открыть мобильное меню}{act:lobby:header:mobile-menu}
\uibutton{Кнопка меню (``бургер'')}{ui:lobby:btn-mobile-menu}
\paragraph{Действие}
Открывает/закрывает мобильную навигацию (на малых экранах).

\component{Заголовок лобби}{cmp:lobby:title}

\paragraph{Назначение}
Отображает имя текущего лобби/сессии.

\paragraph{Текущее отображение}
В примере в заголовке отображается имя \texttt{aaa}.

\component{Участники лобби}{cmp:lobby:participants}

\paragraph{Назначение}
Показывает подключённых участников и их состояние (готовность, время подключения, выбранный персонаж), а также админские действия.

\screenshotPlaceholder{Список участников}

\component{Карточка участника}{cmp:lobby:participants:card}

\paragraph{Назначение}
Информация о конкретном участнике (ник, статус, время подключения, выбранный персонаж).

\action{Кикнуть участника}{act:lobby:participants:kick}
\uibutton{Кнопка ``Кикнуть''}{ui:lobby:participants:btn-kick}
\paragraph{Действие}
Удаляет участника из лобби (доступность действия зависит от прав пользователя).

\action{Сбросить персонажа участника}{act:lobby:participants:reset-character}
\uibutton{Кнопка ``Сбросить персонажа''}{ui:lobby:participants:btn-reset-character}
\paragraph{Действие}
Сбрасывает привязку выбранного персонажа у участника (например, делает персонажа свободным).

\component{Персонажи}{cmp:lobby:characters}

\paragraph{Назначение}
Список персонажей, доступных в лобби, и их текущее состояние (свободен/занят/и т.п.).

\screenshotPlaceholder{Список персонажей}

\component{Счётчик свободных}{cmp:lobby:characters:free-counter}
\paragraph{Назначение}
Показывает количество свободных персонажей относительно общего лимита (например, \texttt{9/10 свободны}).

\component{Карточка персонажа}{cmp:lobby:characters:card}

\paragraph{Назначение}
Одна карточка персонажа (имя, статус, наличие истории). Карточка выглядит кликабельной.

\action{Выбрать персонажа}{act:lobby:characters:select}
\uibutton{Клик по карточке персонажа}{ui:lobby:characters:card-click}
\paragraph{Действие}
Выбирает персонажа для текущего пользователя или открывает просмотр персонажа (точное поведение уточняется по реализации).

\paragraph{Примеры элементов}
В списке есть карточки вида \texttt{Эльф 4509 (Юна)}, \texttt{Эльф 3091 (Сами)} со статусом \texttt{Свободен} и меткой \texttt{Есть история}.
