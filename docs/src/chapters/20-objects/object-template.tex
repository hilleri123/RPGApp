\section{Объекты сценария}
\label{sec:objects}

В этом разделе описаны основные примитивы сценария и ведения игры: что это за сущности, как они связаны и как используются мастером и игроками.

\subsection{Сценарий (Scenario)}
\label{obj:scenario}

\paragraph{Назначение}
Сценарий — контейнер всего контента и настроек конкретной игры: локации, сюжетные точки, персонажи, НПС, предметы, действия, заметки, счётчики и т.д.

\paragraph{Ключевые свойства}
Название, вводный текст (интро), ограничения (например, число игроков), иконка/обложка, привязка к правилам (если используется).

\paragraph{Связи}
Все сущности ниже либо напрямую принадлежат сценарию, либо косвенно связаны через принадлежность объектов сценарию.

\subsection{Сюжетная точка (Story beat)}
\label{obj:story-beat}

\paragraph{Назначение}
Story beat — атомизированный кусочек сценария (сюжетный узел/поворот), удобный для поиска, упорядочивания и построения структуры истории.

\paragraph{Ключевые свойства}
Название, порядок в сценарии, текст для мастера, текст для игроков, флаги видимости (например, показывать/скрывать), поведение «показать один раз», изображение (если используется).

\paragraph{Связи}
Story beat может быть связан с локациями и НПС, а также иметь родительскую сюжетную точку (ветвление/дерево).

\paragraph{Использование в ведении игры}
Когда игроки достигают story beat, мастер может быстро открыть заготовленную экспозицию (см. \ref{obj:scene-exposure}), чтобы сразу показать игрокам подготовленную информацию.

\subsection{Экспозиция сцены (Scene exposure)}
\label{obj:scene-exposure}

\paragraph{Назначение}
Экспозиция сцены — заготовленная выдача информации для игроков в конкретный момент ведения игры: «что показать/описать/предъявить», когда они попали в сюжетную точку или в локацию.

\paragraph{Привязка}
Экспозиция может быть привязана:
\begin{itemize}
  \item к story beat (когда игроки дошли до сюжетной точки);
  \item к локации (когда игроки пришли в конкретное место).
\end{itemize}

\paragraph{Содержимое экспозиции}
Экспозиция может включать набор элементов сцены:
\begin{itemize}
  \item предметы;
  \item НПС;
  \item препятствия/преграды.
\end{itemize}

\paragraph{Использование}
Мастер использует экспозицию как «быстрый показ сцены», чтобы не собирать вручную каждый раз одно и то же и не забывать важные детали.

\subsection{Локация (Location)}
\label{obj:location}

\paragraph{Назначение}
Локация — место в мире сценария, где могут происходить сцены и действия.

\paragraph{Ключевые свойства}
Название, описания для мастера и игроков, флаги видимости, признак стартовой локации, ссылки на карту/изображения, дополнительные данные.

\paragraph{Связи}
Локации могут образовывать иерархию (родитель/дочерние локации).
Локации могут иметь экспозиции (см. \ref{obj:scene-exposure}), которые мастер показывает игрокам при попадании в место.

\subsection{Объекты на карте (Map object polygon)}
\label{obj:map-object-polygon}

\paragraph{Назначение}
Полигоны карты — визуальные области/переходы, которые связывают локации и помогают показывать игрокам карту (зоны, маршруты, переходы).

\paragraph{Ключевые свойства}
Настройки отображения (видимость, заливка/линия, цвет, прозрачность), геометрия полигона, иконки.

\paragraph{Связи}
Полигоны могут указывать на локацию-источник и локацию-цель (например, переход между локациями).

\subsection{НПС (NPC)}
\label{obj:npc}

\paragraph{Назначение}
NPC — неигровой персонаж мира, управляемый мастером.

\paragraph{Ключевые свойства}
Имя, описания для мастера и игроков, флаги (например, враг/мертв/показывать), изображения/иконки, дополнительные данные.

\paragraph{Связи}
НПС может быть связан со story beat и может присутствовать в экспозициях сцены (см. \ref{obj:scene-exposure}).

\subsection{Персонаж игрока (Player character)}
\label{obj:player-character}

\paragraph{Назначение}
Персонаж игрока — сущность, которой управляет конкретный игрок во время сессии.

\paragraph{Ключевые свойства}
Имя, краткое описание, история, изображения/иконки, текущее состояние (в т.ч. произвольные данные), положение (локация).

\paragraph{Связи}
Персонаж может владеть предметами (см. \ref{obj:item-ownership}) и может быть адресатом заметок с доступом (см. \ref{obj:note}).

\subsection{Предмет (Game item)}
\label{obj:game-item}

\paragraph{Назначение}
Предмет — игровая сущность типа «вещь/объект», которая может находиться в локации, у НПС, у персонажа или внутри другого предмета.

\paragraph{Ключевые свойства}
Название, описания для мастера и игроков, видимость, поведение «исчезает при использовании» (если применимо), изображения/иконки, дополнительные данные.

\paragraph{Связи}
Предмет может быть включён в экспозицию сцены (см. \ref{obj:scene-exposure}) и может участвовать во владении (см. \ref{obj:item-ownership}).

\subsection{Владение предметом (Item ownership)}
\label{obj:item-ownership}

\paragraph{Назначение}
Владение предметом описывает, кто чем владеет и в каком количестве.

\paragraph{Кто может быть владельцем}
Владельцем может быть:
\begin{itemize}
  \item персонаж игрока;
  \item НПС;
  \item другой предмет (контейнер: мешок/сундук/инвентарь как предмет, внутри которого лежат другие предметы).
\end{itemize}

\paragraph{Зачем это нужно}
Это даёт единый механизм инвентаря и контейнеров: предметы можно вкладывать в предметы, передавать между персонажами и НПС, учитывать количество.

\subsection{Заметка (Note)}
\label{obj:note}

\paragraph{Назначение}
Заметка — текст, который мастер использует во время ведения игры; заметку можно показывать конкретным игрокам или всем.

\paragraph{Ключевые свойства}
Название, текст, настройки доступа (кому видно), изображения/иконки (если используются), привязка к сценарию.

\paragraph{Использование}
Заметки применяются как «подготовленные сообщения/подсказки/вводные», которые можно выдавать адресно и в нужный момент.

\subsection{Счётчик (Counter)}
\label{obj:counter}

\paragraph{Назначение}
Счётчик — числовой примитив для мастера: прогресс, таймеры, очки, лимиты, состояния и т.п.

\paragraph{Ключевые свойства}
Имя, описание, текущее значение, опциональные минимумы/максимумы, привязка к сценарию и (опционально) к персонажу.

\paragraph{Использование}
Мастер изменяет счётчики по ходу игры, а логика/интерфейс может использовать их для отображения прогресса и контроля условий.

\subsection{Эталоны/фабрики (Template sets)}
\label{obj:templates}

\paragraph{Назначение}
Эталоны (шаблоны) нужны, чтобы во время сессии мастер мог выбрать «заготовку» (персонажа, НПС или предмет) и мгновенно получить копию, не заполняя вручную множество полей.

\paragraph{Поведение}
Выбор эталона создаёт новую сущность в рамках текущего сценария/сессии, копируя преднастроенные поля и данные.
Далее мастер может при необходимости отредактировать копию под текущую ситуацию.

\paragraph{Зачем это нужно}
Это ускоряет импровизацию: добавление нового НПС/предмета/персонажа по ходу игры не ломает темп ведения.
