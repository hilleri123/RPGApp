\section{Игрок}
\label{sec:us-player}

Игрок подключается к игре, настраивает представление себя и персонажа, затем получает от мастера состояние мира и текущие действия.

\userstory{Подключение к игре}{us:p:connect}

Я, как игрок, хочу подключиться к игре и настроить свою персонификацию,
чтобы отличаться от других игроков.

\userstory{Персонаж}{us:p:character}

Я, как игрок, хочу видеть своего персонажа и его состояние,
чтобы понимать, как мне его отыгрывать.

\userstory{Действия}{us:p:actions}

Я, как игрок в игре, хочу видеть описание текущего действия и подтвердить его выполнение,
чтобы избежать иного трактования своего решения.

\userstory{Карта}{us:p:map}

Я, как игрок в игре, хочу просматривать доступную мне карту со всеми пройденными локациями,
чтобы без дополнительных вопросов к мастеру вспоминать путь моего героя.

\userstory{Сцена}{us:p:scene}

Я, как игрок в игре, хочу видеть сцену с её элементами,
чтобы актуализировать свои знания о месте действия и потенциальных возможностях.

\userstory{Карта (общий режим)}{us:c:map}

Я, как участник игры, хочу видеть карту текущей локации,
чтобы иметь более полное понимание описываемого мастером места.

\userstory{Сцена (общий режим)}{us:c:scene}

Я, как участник игры, хочу видеть сцену,
чтобы иметь более полное понимание об описываемой сцене.

\userstory{Хронология}{us:c:timeline}

Я, как участник игры, хочу просматривать историю действий сценария,
чтобы предметно обсуждать или анализировать завершенную игру.
