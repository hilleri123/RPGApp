\section{Создание сценария}\label{sec:standard-actions:create-scenario}

\paragraph{Цель}
Подготовить сценарий для последующего запуска игровой сессии.

\paragraph{Связанные экраны}
Сценарии доступны из верхней навигации приложения (раздел \texttt{/scenarios}). 

\paragraph{Результат}
Появляется сценарий, который можно будет запустить (перейти к действиям из раздела \ref{sec:standard-actions:run-game}). 

\paragraph{Примечание}
Если в текущей сборке часть шагов по наполнению сценария (сцены, локации, заметки и т.п.) ещё не выделена в отдельные формы, их можно фиксировать уже во время проведения сессии через правую панель сущностей и нижний блок логов.
