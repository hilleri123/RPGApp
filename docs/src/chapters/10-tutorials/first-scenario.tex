\section{Первый сценарий: превращаем текст в объекты}\label{tut:first-scenario}

\paragraph{Цель}
Понять, как исходный наративный текст сценария преобразуется в объекты в системе (локации, NPC, story beat, препятствия/проверки), чтобы затем было удобно вести игру.

\paragraph{Пример фрагмента сценария}
\begin{quote}
Герои попадают в замок, где их встречает дворецкий.
Он говорит, что можно ходить куда хочешь, кроме двери.
Дверь можно открыть отмычками со сложностью 14.
Дворецкого можно убедить пропустить к двери; сложность убеждения 12.
\end{quote}

\paragraph{Как читать этот текст}
В тексте обычно смешаны: (1) окружение, (2) персонажи, (3) сцены/события, (4) правила и проверки.
В системе это удобнее держать отдельно, чтобы не терять детали и быстро возвращаться к ним во время игры.

\subsection{Разбор на объекты}\label{tut:first-scenario:decomposition}

\paragraph{Шаг 1. Найди сущности}
Из примера выше выделяем:
\begin{itemize}
\item Локация: \textbf{Замок}.
\item NPC: \textbf{Дворецкий}.
\item Важный объект/точка интереса: \textbf{Запретная дверь}.
\end{itemize}

\paragraph{Шаг 2. Найди ``что происходит''}
Это и есть заготовка story beat:
\begin{itemize}
\item Событие: ``Встреча в замке и установка запрета на дверь''.
\item Участники: дворецкий, герои.
\item Контекст: замок (у входа/в холле).
\end{itemize}

\paragraph{Шаг 3. Найди проверки/препятствия}
Проверка = ``условие + метод + сложность + результат''.
В тексте две проверки:
\begin{itemize}
\item Взлом двери отмычками: сложность 14.
\item Убедить дворецкого: сложность 12.
\end{itemize}

\subsection{Заполнение на сайте}\label{tut:first-scenario:fill-on-site}

\subsubsection{Создай локацию ``Замок''}\label{tut:first-scenario:create-location}

\paragraph{Где}
Раздел ``Сценарии'' $\rightarrow$ открой сценарий $\rightarrow$ добавь локацию.

\paragraph{Что заполнить (минимум)}
\begin{itemize}
\item \textbf{Name}: Замок.
\item \textbf{Description}: ``Холл замка. Вас встречает дворецкий. Можно ходить куда угодно, кроме запретной двери.''
\item \textbf{Сцена}: отметить, что существует дверь с запретом (ещё без механики).
\end{itemize}

\paragraph{Привязка препятствия к локации}
Добавь препятствие в сцену локации:
\begin{itemize}
\item Название: ``Запретная дверь''.
\item Метод: ``Взлом отмычками''.
\item Сложность: 14.
\item Успех: дверь открывается.
\item Провал: дверь не поддаётся, есть риск шума/подозрений (формулировку выбери по стилю игры).
\end{itemize}


\subsubsection{Создай NPC ``Дворецкий''}\label{tut:first-scenario:create-npc}

\paragraph{Где}
Внутри сценария добавь NPC.

\paragraph{Что заполнить (минимум)}
\begin{itemize}
\item \textbf{Name}: Дворецкий.
\item \textbf{Description}: ``Вежливый, настойчивый, следит за соблюдением правил дома.''
\item \textbf{Крючок для общения}: ``Почему нельзя к двери? Что он скрывает/чего боится?''
\end{itemize}


\subsubsection{Создай Story beat ``Встреча у входа''}\label{tut:first-scenario:create-story-beat}

\paragraph{Зачем}
Story beat фиксирует, что именно происходит сейчас, и держит ``под рукой'' всё важное: где мы, кто участвует, что игроки слышат, какие есть варианты действий.

\paragraph{Что заполнить (минимум)}
\begin{itemize}
\item Название: ``Встреча в замке''.
\item Ссылка на локацию: Замок.
\item Ссылки на участников: Дворецкий.
\item Текст мастеру: ``Дворецкий приветствует героев, разрешает свободно ходить, но запрещает приближаться к двери.''
\end{itemize}

\paragraph{Препятствие в сцену story beat}
Добавь социальную проверку как препятствие, связанное именно с этой сценой:
\begin{itemize}
\item Название: ``Убедить дворецкого''.
\item Метод: ``Убеждение''.
\item Сложность: 12.
\item Успех: дворецкий пропускает/отвлекается/даёт ключевую информацию.
\item Провал: усиливает запрет, зовёт охрану, начинает следить (выбери мягкий или жёсткий вариант).
\end{itemize}
